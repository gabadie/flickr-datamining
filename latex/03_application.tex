\chapter{Application}

   % TODO : Rajouter + de texte autour du contexte de ce projet, et sur les technologies utilisées dans cette première partie : Knime, ..%
    But~: trouver des points d’intérêts.

\section{Familiarisation avec les données}
    % TODO : Rajouter + de détails => format des données, origine, contexte, colonnes du CSV, ... %
    Découverte des colonnes disponibles, de leur type (attendu)
    Nous choisissons d'éliminer les données concernant la date d'upload (la date de prise de la photo étant plus pertinente)


\section{Préparation et nettoyage des données}

    \begin{description}
        \item[Paramètres de lecture du fichier CSV~:] choix du séparateur (tabulation, dans notre cas), autorisation des \textit{shortlines}, ...
        \item[Type des colonnes~:] certaines colonnes étaient considérées comme ``String'' alors qu'elles étaient censées être des ``Double''. Ceci est dû à des valeurs parasites (la fameuse valeur de latitude ``trolilol'', notamment).
        \item[Unicité des IDs~:] Il apparaît que l'ID des photos n'est pas unique. Ceci peut être dû au fait que la récupération des données se fait en parallèle, et que les informations d'une même photo peuvent être récupérées plusieurs fois.
        \item[Deux solutions sont possibles~:] en partant de l'hypothèse que ces duplicatas représentent les mêmes informations, on peut arbitrairement garder un élément par id de photo. Une deuxième solution, plus rigoureuse, est de créer un nouvel id garantissant l'unicité des informations (càd l'ensemble des colonnes pertinentes: id de la photo, id de l'utilisateur, date, tags et légende).
        \item[~:]
    \end{description}


\section{Méthodes de clustering}

    \subsection{Clustering hiérarchique}
    Cette méthode, bien qu'utile quand on ne connaît pas le nombre de clusters attendu, ne passe pas à l'échelle~: son exécution sur la totalité du jeu de donnée provoque un dépassement de mémoire.
    Nous avons malgré tout réalisé un échantillonnage aléatoire (de 1,000 lignes) de notre jeu de données sur lequel on a effectué un \textit{clustering hiérarchique} afin d'avoir une idée sur le nombre de clusters qui pouvaient être extraits. Mais même sur un jeu de données aussi restreint (par rapport aux données originales), le résultat était difficilement exploitable (car très dense), et le fait que les données exploitées représentaient moins de 1\% des données originales faisait que ce clustering était fort instable (structure variant sur des samplings différents).

    \subsection{K-Means}


    \subsection{DBScan}

